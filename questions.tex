\documentclass[12pt]{article}
\usepackage[utf8]{inputenc}
\usepackage[english,russian]{babel}
\usepackage{graphicx}
\usepackage{amsmath}
\usepackage{amsthm}
\usepackage{caption}
\usepackage{dsfont}
\usepackage{tikz}
\usepackage{amssymb}
\usepackage{subcaption}
\usepackage{imakeidx}
\usepackage{hyperref}
\usepackage[russian]{cleveref}
\usepackage[a4paper,left=15mm,right=15mm,top=30mm,bottom=20mm]{geometry}
\parindent=0mm
\parskip=3mm
\DeclareRobustCommand{\divby}{%
\mathrel{\text{\vbox{\baselineskip.65ex\lineskiplimit0pt\hbox{.}\hbox{.}\hbox{.}}}}%
}

\makeindex
\pagestyle{empty}
\title{Задачки деда. Старый хитрее, стырый мудрее. Старый жизнь блять отжил.}
\author{ОПГ Овчинников}
\date{\today}
\begin{document}
\maketitle
\large

\begin{center}
    Артемий
\end{center}

1. \textbf{Асимптотика. Обозначения $o$, $\mathcal{O}$, $\Theta$, $\Omega$, $\omega$.}\\
Асимптотика - поведение функции при стремлении аргумента к бесконечности.\\
Пусть $f(n)$ и $g(n)$ - две функции, которые стремяться к бесконечности, тогда:
\begin{itemize}
    \item $f(n) = o(g(n))$, если $g$ доминирует над $f$ асимптотически, то есть $\forall(С > 0) \exists N : \forall(n \in N)\ |f(n)| < C|g(n)|$
    \item $f(n) = \mathcal{O}(g(n))$, если $f$ ограничена сверху функцией $g$ асимптотически, то есть $\forall(С > 0) \exists N : \forall(n \in N)\ |f(n)| \leq C|g(n)|$
    \item $f(n) = \Theta(g(n))$, если $f$ ограничена снизу и сверху функцией 
    $g$ асимптотически, то есть $\forall(С_{1} > 0), (C_{2} > 0) \ \exists N : \forall(n \in N)\ C_{1}|g(n)| \leq |f(n)| \leq C_{2}|g(n)|$
    \item $f(n) = \Omega(g(n))$, если $f$ ограничена снизу функцией 
    $g$ асимптотически, то есть $\forall(С > 0) \exists N : \forall(n \in N)\ |f(n)| \geq C|g(n)|$
    \item $f(n) = \omega(g(n))$, если $f$ доминирует над $g$ асимптотически, то есть $\forall(С > 0) \exists N : \forall(n \in N)\ |f(n)| > C|g(n)|$
\end{itemize}

Прекрасная лекция на 30-40 минут, в которой эти штуки объясняют в том числе графически, что намного легче для понимания, даже необязательно всё смотреть: \href{https://www.youtube.com/watch?v=dpdimxKMMTc}{\textit{ТЫК}}.

2. \textbf{Основные свойства асимптотики. Асимптотика многочлена.}\\
Тут если первый билет понять, проблем быть не должно.\\
Транзитивность:
\begin{itemize}
    \item $f(n) = \Theta (g(n)) \land g(n) = \Theta (h(n)) \Rightarrow f(n) = \Theta (h(n))$
    \item $f(n) = \mathcal{O} (g(n)) \land g(n) = \mathcal{O} (h(n)) \Rightarrow f(n) = \mathcal{O} (h(n))$
    \item $f(n) = \Omega (g(n)) \land g(n) = \Omega (h(n)) \Rightarrow f(n) = \Omega (h(n))$
    \item $f(n) = o(g(n)) \land g(n) = o(h(n)) \Rightarrow f(n) = o(h(n))$
    \item $f(n) = \omega (g(n)) \land g(n) = \omega (h(n)) \Rightarrow f(n) = \omega (h(n))$
\end{itemize}
Рефлексивность:\\
$f(n) = \Theta (g(n)) \Rightarrow f(n) = \mathcal{O}(g(n))$; $f(n) = \Omega(g(n))$\\
Симметричность:\\
$f(n) = \Theta (g(n)) \Rightarrow g(n) = \Theta (f(n))$\\
Перестановочная симметрия:\\
$f(n) = \mathcal{O}(g(n)) \Leftrightarrow g(n) = \Omega(f(n))$\\
$f(n) = o(g(n)) \Leftrightarrow g(n) = \omega(f(n))$\\
Хз, на вики нет названия))):
\begin{itemize}
    \item $C \cdot o(f(n)) = o(f(n))$\\
    $C \cdot \mathcal{O}(f(n)) = \mathcal{O}(f)$
    \item $o(C \cdot f) = o(f)$\\
    $\mathcal{O}(C \cdot f) = \mathcal{O}(f)$
    \item $o(-f) = o(f)$\\
    $\mathcal{O}(-f) = \mathcal{O}(f)$
    \item $o(f) + o(f) = o(f)$\\
    $o(f) + \mathcal{O}(f) = \mathcal{O}(f) + \mathcal{O}(f) = \mathcal{O}(f)$
    \item $\mathcal{O}(f) \cdot \mathcal{O}(g) = \mathcal{O}(fg)$\\
    $o(f) \cdot \mathcal{O}(g) = o(f) \cdot o(f) = o(fg)$
    \item $\mathcal{O}(\mathcal{O}(f)) = \mathcal{O}(f)$\\
    $o(o(f) = o(\mathcal{O}(f)) = \mathcal{O}(o(f)) = o(f)$
\end{itemize}

Асимтотика многочлена:\\
$P(x)$ - многочлен, тогда $P(x) = \Theta(x^{degP})$ при старшем коэффициенте > 0.\\
$P(x) = a_0 + a_1x + a_2x^2 + ... + a_kx^k$, все слагаемые кроме $a_kx^k$ являются $o(x^{degP})$. Значит вся сумма является $\Theta(x^k)$.

3. \textbf{Определение $o$, $\omega$, $\mathcal{O}$ через пределы.}\\
$f = o(g)$: $\lim\limits_{n\to +\infty} \dfrac{f(n)}{g(n)}=0$\\
\\
$f = \omega(g)$: $\lim\limits_{n\to +\infty} \dfrac{g(n)}{f(n)}=0$\\
\\
$f = \mathcal{O}(g)$: $\varlimsup\limits_{n\to +\infty} \dfrac{f(n)}{g(n)}<\infty$\\

4. \textbf{Отношение доминирования между основными функциями.}\\
Если я правильно понял, то вот:\\
$\log\log n < \log n < n < n\log n < n^2 < n^3 < 2^n < n! < n^n$

5-6. \textbf{Определение однородных линейных рекуррентных соотношений (ОЛРУ). Алгоритм поиска решения. Случай разных корней характеристического уравнения. Случай кратных корней характеристического уравнения.}\\
Определение:\\
Последовательность $u_1, u_2, ..., u_n, ...$ называется однородной линейной рекуррентной последовательностью порядка $k$, если существуют натуральное число $k$ и числа $a_1, a_2, ..., a_k$ (действительные или комплексные, причем $a_k \neq 0$) такие, что для любого $n$ справедливо $u_{n+k} = a_1u_{n+k-1} + a_2u_{n+k-2} + ... + a_ku_n$ ($\leftarrow$ это и называется однородным линейным рекуррентным соотношением порядка $k$).\\
Алгоритм решения (далее будут примеры, на которых понятнее):
\[
f_{n-k} = a_1f_{n+k-1} + ... + a_kf_n
\]
\[
r^k = a_1r^{k-1} + ... + a_k
\]
\[
r^k - a_1r^{k-1} - ... - a_k = 0
\]
Далее находим корни получившегося уравнения и считаем кратность $L$ каждого корня\\
Общее решение:
\[
f_{n+k} = S_1 + S_2 + ... + S_k
\]
\[
\begin{array}{c}
    \text{причём}\\
    S_i = (c_{i_1} + c_{i_2}n + c_{i_3}n^2 + ... + c_{i_L}n^{L-1}) \cdot r_i^n\\
    r_i^n \leftarrow \text{сам корень}
\end{array}
\]
Ищем частное решение (получаем $k$ первых элементов и составляем систему):
\[
f_0 = v_0, f_1 = v_1, \dots, f_k = v_k
\]
\[
 \begin{cases}
   v_0 = \sum_{j=1}^k S_j 
   \\
   v_1 = \sum_{j=1}^k S_j
   \\
   \dots
   \\
   v_k = \sum_{j=1}^k S_j
 \end{cases}
\]
Проверяем, совпадают ли ответы на первых членах, сравнивая с  изначальной рекуррентной формулой.\\
Примеры:\\
\textit{С кратностью корней}
\[
f(n-2) = -6f(n-1) - 9f(n), \text{ где } f(0) = 1, f(1) = 5\\
\]
\[
r^2 = -6r - 9
\]
\[
r^2 + 6r + 9 = 0
\]
\[
(r + 3)^2 = 0
\]
где степень 2 - и есть кратность
$r = -3$ (корень)\\
Общее решение:
\[
f(n) = n^0 \cdot c_1 \cdot (-3)^n + n^1 \cdot c_2 \cdot (-3)^n = (-3)^n \cdot (c_1 + n \cdot c_2)
\]
Степени у $n$ соответствуют кратности $[0, L-1] \cap \mathbb{Z}$ \\
В данном случае $[0, 1] \cap \mathbb{Z}$, то есть $n^0$ и $n^1$ у соответствующих корней\\
Напомним, что $f(0) = 1$, $f(1) = 5$\\
\[
 \begin{cases}
   (-3)^0 \cdot (c_1 + 0 \cdot c_2) = 1
   \\
   (-3)^1 \cdot (c_1 + 1 \cdot c_2) = 5
 \end{cases} 
 \Rightarrow
 \begin{cases}
    c_1 = 1\\
    c_2 = -\frac{8}{3}
 \end{cases} 
\]
Частное решение:
\[
f(n) = (-3)^n \cdot (1 - n \cdot \frac{8}{3})
\]
Проверка:\\
a) Рекуррент\\
$f(2) = -6 \cdot 5 - 9 = -39$\\
$f(3) = -6 \cdot (-39) - 9 \cdot 5 = 189$\\
$f(4) = -6 \cdot 189 - 9 \cdot (-39) = -783$\\
б) Явная формула\\
$f(2) = (-3)^2 \cdot (1 - 2 \cdot \frac{8}{3}) = -39$\\
$f(3) = (-3)^3 \cdot (1 - 3 \cdot \frac{8}{3}) = 189$\\
$f(4) = (-3)^4 \cdot (1 - 4 \cdot \frac{8}{3}) = -783$\\
\\
\textit{Разные корни + как избавиться от свободного члена (просто число без $f(...)$)}
\[
f(n+2) - 12f(n+1) + 27f(n) = 8,\text{ где } f(0) = 4, f(1) = -2
\]
Чтобы избавиться от свободного члена (8) и всё равно получить верное решение, нужно увеличить каждое слагаемое в индексе на 1, затем вычесть из полученного выражения изначальное:\\
\[
-
 \begin{array}{c}
   f(n+3) - 12f(n+2) + 27f(n+1) = 8
   \\
   f(n+2) - 12f(n+1) + 27f(n) = 8
 \end{array}
\]
Получаем:
\[
    f(n+3) - 13f(n+2) + 39f(n+1) - 27f(n) = 0
\]
$r^3 - 13r^2 + 39r - 27 = 0$\\
$(r - 1)(r - 3)(r - 9) = 0$\\
$r_1 = 1$, $r_2 = 3$, $r_3 = 9$\\
Никакой кратности (одинаковых корней) у нас нет, значит у каждого слагаемого в общем решении будет множитель $n^0 = 1$\\
\[
    f(n) = c_1 + c_2 \cdot 3^n + c_3 \cdot 9^n
\]
\[
 \begin{cases}
   f(0) = c_1 + c_2 + c_3 = 4\\
   f(1) = c_1 + 3c_2  + 9c_3 = -2\\
   f(2) = c_1 + 9c_2 + 81c_3 = -124
 \end{cases}
\]
Решаем систему (photomath тема) и получаем, что $c_1 = \frac{1}{2}$, $c_2 = \frac{17}{3}$, $c_3 = -\frac{13}{6}$\\
$f(n) = \frac{1}{2} + \frac{17}{3} \cdot 3^n - \frac{13}{6} \cdot 9^n$\\
Проверка:\\
a) Рекуррент\\
$f(2) = 8 + 12f(1) - 27f(0) = 8 + 12 \cdot (-2) - 27 \cdot 4 = -124$\\
$f(3) = 8 + 12f(2) - 27f(1) = 8 + 12 \cdot (-124) - 27 \cdot (-2) = -1426$\\
$f(4) = 8 + 12f(3) - 27f(2) = 8 + 12 \cdot (-1426) - 27 \cdot (-124) = -13756$\\
б) Явная формула\\
$f(2) = \frac{1}{2} + \frac{17}{3} \cdot 3^2 - \frac{13}{6} \cdot 9^2 = -124$\\
$f(3) = \frac{1}{2} + \frac{17}{3} \cdot 3^3 - \frac{13}{6} \cdot 9^3 = -1426$\\
$f(4) = \frac{1}{2} + \frac{17}{3} \cdot 3^4 - \frac{13}{6} \cdot 9^4 = -13756$\\


7. \textbf{Числа Фибоначчи. Определение, формула в замкнутом виде.}\\
Числа Фибоначчи - элементы числовой последовательности, в которой первые два числа равны 1 и 1, а каждое последующие число равно сумме двух предыдущих; однородное линейное рекуррентное соотношение второго порядка\\
$F_n = F_{n-1} + F_{n-2}$\\
Формаула в замкнутом виде:\\
$F_n = \dfrac{\Big ( \dfrac{1 + \sqrt{5}}{2} \Big )^n - \Big ( \dfrac{1 - \sqrt{5}}{2} \Big )^n}{\sqrt{5}} = \dfrac{\varphi^n - (-\varphi)^{-n}}{\varphi - (-\varphi^{-1})} = \dfrac{\varphi^n - (-\varphi)^{-n}}{2\varphi - 1}$\\
$\varphi = \dfrac{1 + \sqrt{5}}{2}$ - золотое сечение\\
Формула без золотого сечения:\\
$F_n = \dfrac{1}{2^n\sqrt{5}} \big ( (1 + \sqrt{5})^n - (1 - \sqrt{5})^n \big )$

8. \textbf{Определение неоднородных линейных рекуррентных соотношений (НЛРУ). Общий алгоритм поиска решения.}\\
Оперделение:\\
Последовательность $u_1, u_2, ..., u_n, ...$ называется неоднородной линейной рекуррентной последовательностью порядка $k$, если существует натуральное число $k$, числа $a_1, a_2, ..., a_k$ (действительные или комплексные, причем $a_k \neq 0$) и функция $b_n = f(n)$ такие, что  $u_{n+k} = a_1u_{n+k-1} + a_2u_{n+k-2} + ... + a_ku_n + b_n$ ($\leftarrow$ это и называется неоднородным линейным рекуррентным соотношением порядка $k$).\\
Алгоритм решения (примеры для лучшего понимая далее в билетах 10, 12, 13):\\
$g(n) = a_1n^k + ... + a_{k+1}$\\
1) Решаем ОЛРУ\\
2) Рассмотрим частное решение нерекуррентной части
\[
    f_p(n) = a_1n^k + ... + a_{k+1} = g(n)
\]
Подставим данное $f_p(n)$ в исходное реккурентное уравнение
\[
    f_a(f_p(n)) + f_{a-1}(f_p(n)) + ... + f_{a-k}(f_p(n)) = g(n)
\]
Соответствующие члены и будут членами частного решения нерекуррентной части.\\
Если корень решения ОЛРУ равен 1, то при подстановке $f_p(n)$ в рекуррентное уравнение у нас оно может не сойтись $\Rightarrow$ повышаем степени членов в $f_p(n)$ (и будем так перебирать, пока не получится корректный ответ)\\\\
3) Общим решением такого НЛРУ будет общее решение ОЛРУ + частное решение нереккурентной части

10. \textbf{Поиск частного решения НЛРУ при функции-константе.}\\
См билеты 5-6, а именно пример \textit{Разные корни + как избавиться от свободного члена (просто число без $f(...)$)}

12. \textbf{Поиск частного решения НЛРУ при функции-многочлене.}\\
Пример:\\
\[
f(n+2) - 2f(n+1)  + f(n) = -9n - 1
\]
1) 
\[
r^2 - 2r + 1 = 0\\
\]
\[
\begin{array}{cc}
    (r - 1)^2 = 0 & r = 1 (!)\\
\end{array}
\]
\[
f_r(n) = c_1 + n \cdot c_2
\]
2) 
\[
f_p(n) = an^2+ bn
\]
\[
(a(n+2)^2 + b(n+2)) - 2(a(n+1)^2 + b(n+1)) + (an^2 + bn) = -9n - 1
\]
\[
    2a = -9n - 1 \text{-- грусть печаль, нам такое не надо(}
\]
\[
f_p(n) = an^3+ bn^2
\]
\[
(a(n+2)^2 + b(n+2)^2) - 2(a(n+1)^3 + b(n+1)^2) + (an^3 + bn^2) = -9n - 1
\]
\[
6a_n + (6a + 2b) = -9n - 1
\]
\begin{equation*}
 \begin{cases}
   $6a = -9$
   \\
   $6a + 2b = -1$
 \end{cases}
\end{equation*}
Получаем $a = -1.5$, $b = 4$\\
\[f_p(n) = -1.5n^3 + 4n^2\]
3) Общая формула:
\[f(n) = f_r(n) + f_p(n) = c_1 + nc_2 - 1.5n^3 + 4n^2\]
(Далее можно получить частную формулу и проверить её, как мы это делаем в ОЛРУ)

\begin{center}
    Игорь (платница межбак)
\end{center}

13. \textbf{Поиск частного решения НЛРУ при функции-экспоненте.}\\
$g(n)$ - экспонента = $d \cdot (\alpha)^n$\\
1) Вновь отдельно решаем ОЛРУ, отбросив нереккурентную часть и получаем $f_r(n)$\\
2) В этот раз начинаем с $f_p(n) = d \cdot (\alpha)^n$ и, при необходимости, по аналогии с предыдущим алгоритмом перебираем варианты, умножением на $n$ (обычно, если один из корней ОЛРУ равен $\alpha$, то надо умножить на $n$ b перебирать с $f_p(n) = d \cdot (\alpha)^n
\cdot n$).\\
Так получаем частное решение нерекуррентной части для НЛРУ с экспонентой.\\
3) Вновь:
\[f(n) = f_r(n) + f_p(n)\]
Пример:\\
\[f(n+2) + 19f(n+1) + 90f(n) = 2 \cdot (-10)^{n+2}\]
1) 
\[r^2 + 19r + 90 = 0\]
\[(r + 10)(r + 9) = 0\]
\[
\begin{cases}
    r_1 = -10\\
    r_2 = -9    
\end{cases} \Rightarrow f_r(n) = (-10)^nc_1 + (-9)^nc_2
\]
2) 
\[f_p(n) = d \cdot (-10)^{n+2} \cdot n\]
\[
(d \cdot (-10)^{n+4} \cdot (n + 2)) + 19(d \cdot (-10)^{n+3} \cdot (n + 1)) + 90(d \cdot (-10)^{n+2} \cdot n) = 2 \cdot (-10)^{n+2}
\]
Поделим на $(-10)^{n+2}$\\
\[
(d \cdot (-10)^{2} \cdot (n + 2)) + 19(d \cdot (-10) \cdot (n + 1)) + 9(d \cdot n) = 2
\]
\[
d = 0,2 \Rightarrow f_p(n) = 0,2 \cdot (-10)^{n+2} \cdot n
\]
3) $f(n) = (-10)^nc_1 + (-9)^nc_2 + 0,2(-10)^{n+2}n$\\
(Далее можно полчить частную формулу и проверить её, как мы это делаем в ОЛРУ)

15. \textbf{Решение рекуррентных уравнений подстановкой на примере T(n) = aT(n/m) + bn, T(1) = b. «Угадывание» итогового решения.}

17. \textbf{Рекуррентные соотношения: основная теорема. Формулировка.}\\
Мастер теорема (теорема о просто рекуррентном соотношении)\\
Пусть $T(n) = aT(\frac{n}{b}) + f(n)$, где $f(n) = n^c$.\\ При этом  $a > 0, b > 1, c \geq 0$. Определим глубину рекурсии $k = \log_{b}n$. \\
\[
\begin{cases}
T(n) = aT(\frac{n}{b}) + f(n)\\
f(n) = n^c
\end{cases}
\]
Тогда верно одно из трёх:
\[
\left\{
\begin{array}{ll}
   T(n) = \Theta(a^k) = \Theta(n^{\log_ba}) & a > b^c\\
   T(n) = \Theta(f(n)) = \Theta(n^c) & a < b^c
   \\
   T(n) = \Theta(k \cdot f(n)) = \Theta(n^c\log n) &  a = b^c
 \end{array}
\right.
\]

18. \textbf{Оценка сумм через интеграл. Основная идея, оценки сверху и снизу.}\\

19. \textbf{Асимптотическая оценка $\sum_{i=0}^{n-1} i^k$, $\sum_{i=0}^{n-1} \log i$, $\sum_{i=0}^{n-1} i\log i$.}

21. \textbf{Алгоритм Евклида поиска НОД.}\\

22. \textbf{Расширенный алгоритм Евклида.}\\

23. \textbf{Понятие мультипликативного обратного. Поиск с помощью алгоритма Евклида.}\\

24. \textbf{Число бинарных деревьев с n вершинами в рекурсивной и нерекурсивной форме. Идея решения через производящие функции.}\\

25. \textbf{Понятие правильной скобочной последовательности (ПСП). Число ПСП длиной 2n.}\\

\begin{center}
    Эрнест
\end{center}

26. \textbf{Понятие пути Дика. Количество путей Дика длиной 2n.}

28. \textbf{Понятие сложности алгоритма. Лучший, худший, средний случай.}

29. \textbf{Ханойская башня. Алгоритм. Доказательство принадлежности к классу экспоненциальных за- дач.}

30. \textbf{Ханойская башня. Оценка сложности через решение рекуррентного уравнения.}

32. \textbf{Вычисление веса двоичного вектора. Полный перебор. Оценка сложности.}

33. \textbf{Вычисление веса двоичного вектора. Предвычисление. Оценка сложности.}

34. \textbf{Вычисление веса двоичного вектора со сложностью $\mathcal{O}$($W$(x)). Особенности реализации.}

36. \textbf{Задача коммивояжёра. Формулировка, вариации условий, матрица стоимости.}

37. \textbf{Формализация постановки задачи. Описание решения задачи коммивояжёра как перечисления гамильтоновых циклов.}

38. \textbf{Формализация постановки задачи. Описание решения задачи коммивояжёра как поиска гамильтонова цикла.}

\begin{center}
    Амина
\end{center}

39. \textbf{Особенности асимптотической оценки сложности алгоритма.}

Для асимптотики не важна константа и в реальной ситуации лучший асимптотически алгоритм может оказаться хуже более "долгого"

40. \textbf{Особенности точной оценки реализации алгоритма.}

41. \textbf{Определение методов частных целей, подъёма вверх, отрабатывания назад.}

42. \textbf{Применение методов разработки алгоритмов на примере задачи о джипе.}

43. \textbf{Отрабатывание назад. Задача о спичках.}

44. \textbf{Подъём вверх. Задача о миссионерах и каннибалах.}

45. \textbf{Частные цели. Задача о греческом кресте.}

46. \textbf{Подъём вверх. Задача о переливании.}

47. \textbf{Отрабатывание назад. Задача о пиратах.}


\begin{center}
    Гордей
\end{center}


50. \textbf{Подъём вверх. Задача о лёгкой фальшивой монете. Обоснование поиска улучшенного решения.}
Задача: Имеется 25 монет, среди которых, возможно, находится одна фальшивая. 
Фальшивая монета легче остальных, и в нашем распоряжении находятся весы с двумя чашками. 
Требуется определить фальшивую монету или установить, что фальшивых монет нет.

Очевидное решение: разделить на две равные по количеству кучи. (25 - нечетное, значит одна монета не влезет в кучи) \\
1) Кучи равны - монеты в кучах настоящие. Остается проверить монету, которая не влезла в кучи. \\
2) Куча которая легче точно содержит фальшивую монету. Разделим ее на две равные по количеству кучи. Повторим алгоритм. \\

Наглядная визуализация: 
\\TODO вставить картинку (1-Гордей)

Очевидный недостаток такого алгоритма: исходов взвешивания - три варианта, однако равенство используется только при первом и последних двух взвешиваниях. 
В дереве решения нет поддеревьев, соответствующих равенству.

Дополнительные замечания: у дерева 26 вариантов исхода: 25 монет могут быть фальшивыми и 1 вариант когда все настоящие.

\textbf{Улучшение решения:} Так как более, чем троичным деревом взвешиваний быть не может, нужно его макимаьно приблизить к тернарному дереву. 
Будем делить монеты на три примерно равные по количесву кучи. Попробуем реализовать - получилось 3 взвешивания в худшем случае.\\

Наглядная визуализация:
  \\todo вставить картинку (2-Гордей)



51. \textbf{Задача о лёгкой фальшивой монете. Достаточное условие оптимальности решения.}
[продолжение билета №50 (см. выше))]

Поскольку у дерева 26 вариантов исхода: 25 монет могут быть фальшивыми и 1 вариант когда все настоящие, 
и более чем тернарным (с тремя ветками) дерево мы сделать не можем, т.к. исходов взвешивания три, до посленей вершины мы дойдем в худшем случае за 3 хода. 
За 3 хода мы смогли достичь ответа, значит мы нашли оптимальнео решение. Ч.Т.Д.

52. \textbf{Рекурсия. Задача о разбиении.}

54. \textbf{Рекурсия. Задача Иосифа Флавия. Рекуррентное решение.}

Задача: В 67 году н.э. во время иудейско-римской войны Иосиф Флавий и сорок мятежников оказались в ловушке в пещере. 
Предпочтя самоубийство плену, они решили встать в круг и убивать каждого второго из оставшихся в живых. 
Иосиф Флавий хотел остаться в живых и понял, где он должен стоять в кругу, чтобы остаться в живых в череде казней. 
Так он выжил, чтобы рассказать эту легенду. 

В поисках решения: \\
Пусть сейчас в кругу есть сколько то человек. Поскольку с каждым кругом убийств людей в кругу становится в два раза меньше, попробуем свести задачу к рекурентному решению $J(2n) = a\cdot J(n) + b$. Однако число людей в кругу может быть и нечетным. Рассмотрим оба таких варианта:

1) Первый вариант: \\
Сейчас: в кругу $2n$ (четное число) человек. \\
Пронумеруем их от 1 до $2n$ \\
Пусть мы знаем, что из всех человек выживет человек под номером $J(2n)$ \\

После одного круга убийствва: в кругу $n$ человек \\
Если пронумеровать их от 1 до $n$, то можно сказать что из прошлого круга выжли люди с номерами $2n - 1$\\
Пусть для этого круга мы тоже знаем ответ - $J(n)$ \\
Поскольку мы решаем задачу для того же набора людей, и знаем как номер выжившего из круга из $n$ человек сопоставить с номером выжившего 
из круга из $2n$ человек, можно вывести следующее уравнение:
$$J(2n)=2\cdot J(N) - 1$$

2) Второй вариант:
Сейчас: в кругу $2n+1$ (нечетное число) человек \\
Пронумеруем их от 1 до $2n+1$ \\
Пусть мы знаем, что из всех человек выживет человек под номером $J(2n+1)$ \\

После одного круга убийств: в кругу останется $n$ человек.
Их номера можно соотнести с предидущим кругом так: $2n+1$
Аналогично задача сводится к:
$$J(2n+1) = 2\cdot J(n) + 1$$

Значит, решение данной задачи можно записать следующим рекурентым отношением:\\
\[
\begin{cases}
J(2n+1) = 2\cdot J(n) + 1
\\
J(2n)=2\cdot J(N) - 1
\\
J(1) = 1
\end{cases}
\]

56. \textbf{Умножение однократной декомпозицией: идея, время работы.}

57. \textbf{Умножение рекурсивной декомпозицией: идея, время работы.}

58. \textbf{Связь умножения чисел и многочленов. Умножение за $\mathcal{O}$($n^2$).}

59. \textbf{Алгоритм Карацубы: идея, время работы.}\\
\url{https://youtu.be/SlMVJOYLmkc?t=1714}
смотреть до 30:07\\
Декомпозируем: разобьем числа на 2 части по $n/2$ бит.
\[
  \left\{
  \begin{array}{lccc}
    U = [a, b] & \text{где} & a = (u_{n-1}, \dots, u_{n/2}) & b = (u_{\frac{n}{2}-1}, \dots, u_0)\\
    U = [c, d] & \text{где} & a = (v_{n-1}, \dots, v_{n/2}) & b = (v_{\frac{n}{2}-1}, \dots, v_0)\\
  \end{array}
  \right.
\] 
Из этого следует, что 
\[
  \begin{array}{l}
    U = a \cdot 2^{n/2} + b\\
    V = c \cdot 2^{n/2} + d
  \end{array}
\]
\[
\Rightarrow
    U \cdot V = (a \cdot 2^{n/2} + b)(c\cdot 2^{n / 2} + d)\\
\]
\[
    U \cdot V = ac \cdot 2^n + (ad + bc) \cdot 2^{n/2} + bd
\]


60. \textbf{Понятие кодов, сохраняющих разность.}

61. \textbf{Композиция. Построение кодов, сохраняющих разность, с квадратичным увеличением слов.}

62. \textbf{Композиция.Построениекодов,сохраняющихразность,сэкспоненциальнымувеличениемслов.}

\begin{center}
    Остатки
\end{center}

63. \textbf{Эвристика. Метод ближайшего соседа в задаче коммивояжёра.}

64. \textbf{Эвристика. Задача о расписании процессоров.}

\end{document}
